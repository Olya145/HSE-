\documentclass[a4paper,12pt,leqno]{article}

%%% Работа с русским языком
\usepackage{cmap}					% поиск в PDF
\usepackage{mathtext} 				% русские буквы в формулах
\usepackage[T2A]{fontenc}			% кодировка
\usepackage[utf8]{inputenc}			% кодировка исходного текста
\usepackage[english,russian]{babel}	% локализация и переносы

%%% Дополнительная работа с математикой
\usepackage{amsmath,amsfonts,amssymb,amsthm,mathtools} % AMS
\usepackage{icomma} % "Умная" запятая: $0,2$ --- число, $0, 2$ --- перечисление

%% Номера формул
%\mathtoolsset{showonlyrefs=true} % Показывать номера только у тех формул, на которые есть \eqref{} в тексте.
%\usepackage{leqno} % Нумерация формул слева

%% Свои команды
\DeclareMathOperator{\sgn}{\mathop{sgn}}

%% Перенос знаков в формулах (по Львовскому)
\newcommand*{\hm}[1]{#1\nobreak\discretionary{}
{\hbox{$\mathsurround=0pt #1$}}{}}

%%% Работа с картинками
\usepackage{graphicx}  % Для вставки рисунков
\graphicspath{{C:/latex/}}  % папки с картинками
\setlength\fboxsep{3pt} % Отступ рамки \fbox{} от рисунка
\setlength\fboxrule{1pt} % Толщина линий рамки \fbox{}
\usepackage{wrapfig} % Обтекание рисунков текстом

%%% Работа с таблицами
\usepackage{array,tabularx,tabulary,booktabs} % Дополнительная работа с таблицами
\usepackage{longtable}  % Длинные таблицы
\usepackage{multirow} % Слияние строк в таблице

%%% Теоремы
\theoremstyle{plain} % Это стиль по умолчанию, его можно не переопределять.
\newtheorem{theorem}{Теорема}[section]
\newtheorem{proposition}[theorem]{Утверждение}
 
\theoremstyle{definition} % "Определение"
\newtheorem{corollary}{Следствие}[theorem]
\newtheorem{problem}{Задача}[section]
 
\theoremstyle{remark} % "Примечание"
\newtheorem*{nonum}{Решение}

%%% Программирование
\usepackage{etoolbox} % логические операторы
 

%%% Страница
\usepackage{extsizes} % Возможность сделать 14-й шрифт
\usepackage{geometry} % Простой способ задавать поля
	\geometry{top=25mm}
	\geometry{bottom=35mm}
	\geometry{left=35mm}
	\geometry{right=20mm}
 
\usepackage{setspace} % Интерлиньяж
%\onehalfspacing % Интерлиньяж 1.5
%\doublespacing % Интерлиньяж 2
%\singlespacing % Интерлиньяж 1

\usepackage{lastpage} % Узнать, сколько всего страниц в документе.

\usepackage{soul} % Модификаторы начертания

\usepackage{hyperref}
\usepackage[usenames,dvipsnames,svgnames,table,rgb]{xcolor}
\hypersetup{				% Гиперссылки
    unicode=true,           % русские буквы в раздела PDF
    pdftitle={Заголовок},   % Заголовок
    pdfauthor={Автор},      % Автор
    pdfsubject={Тема},      % Тема
    pdfcreator={Создатель}, % Создатель
    pdfproducer={Производитель}, % Производитель
    pdfkeywords={keyword1} {key2} {key3}, % Ключевые слова
    colorlinks=true,       	% false: ссылки в рамках; true: цветные ссылки
    linkcolor=black,          % внутренние ссылки
    citecolor=black,        % на библиографию
    filecolor=magenta,      % на файлы
    urlcolor=blue           % на URL
}

\usepackage{csquotes} % Еще инструменты для ссылок

\usepackage[style=authoryear,maxcitenames=2,backend=biber,sorting=nty]{biblatex}

\usepackage{multicol} % Несколько колонок

\usepackage{tikz} % Работа с графикой
\usepackage{pgfplots}
\usepackage{pgfplotstable}

\author{Кустовская Ольга, БКЛ 171}
\title{James Alan Hetfield}
\date{\today}

\begin{document} % конец преамбулы, начало документа

\maketitle

\tableofcontents

Джеймс Алан Хэтфилд (англ. James Alan Hetfield; род. 3 августа 1963, Дауни, Калифорния США) — американский рок-музыкант, вокалист и гитарист. Лидер и один из основателей группы Metallica. C 2011 года занимает 87-е место в списке величайших гитаристов всех времён по версии журнала Rolling Stone. У него есть своя манера держать медиатор тремя пальцами. Он обладает сильным, запоминающимся голосом и особой манерой общения с публикой. Несмотря на то, что Джеймс выступает в Metallica как ритм-гитарист, в некоторых песнях он исполняет и сольные партии гитары \cite{[1]}.

\includegraphics[scale=0.7]{375px-JamesHetfield2012.jpg}

\section{Семья}

Отец Джеймса, Вёрджил (англ. Virgil, ум. в 1996 году), водитель автобуса, покинувший семью, когда Джеймсу было 13, и его мать, Синтия (англ. Cyntia), оперная певица, были последователями религиозного движения Христианская наука Мэри Бэйкер Эдди и, по словам Джеймса, значительная часть его жизни в молодости была связана с христианством \cite{[2]}.

В соответствии со своими верованиями, родители Хэтфилда не одобряли медицину и любой вид медицинского вмешательства, и не отступались от своей веры, даже когда Синтия умирала от рака (Джеймсу тогда было 16 лет) \cite{[3]}. Как рассказывал сам Джеймс, в школе поначалу не было проблем, но когда рассказывали о медицине, он уходил с таких уроков. Он стал отдаляться от всех, когда заметил перешёптывания за его спиной.
 
Смерть матери и сложное отношение к религии позднее стали главными темами многих песен группы Metallica («Mama Said», «Dyers Eve» и «The God That Failed» — о родителях Хэтфилда, а «Until It Sleeps» — о раке).

\section{Музыкальная карьера}

Музицировать Джеймс начал в 9 лет на фортепиано, затем играл на барабанах своего брата Дэйва, и, наконец, перешёл к гитаре. Его первой группой была любительская команда «Obsession». Группа состояла из братьев Veloz, которые играли на бас-гитаре и барабанах, и Джима Арнольда на гитаре. Рон Макговни и Дейв Маррс сидели на чердаке гаража Veloz и щёлкали на пульте управления световыми эффектами. Группа играла известные композиции таких групп как Black Sabbath и Led Zeppelin. После распада Obsession Маррс, Хэтфилд и Макговни продолжали играть вместе.

После переезда в Ла Бреа, Джеймс поступил в школу Brea Olinda и встретился с барабанщиком Джимом Муллиганом. Вскоре появился Хью Таннер, и с ним они создали Phantom Lord. С Хью на соло-гитаре и Джимом на барабанах, Джеймс пел и играл на ритм-гитаре. Группа сменила несколько басистов до того момента, когда Джеймс окончил школу и уехал обратно в Дауни.

В Дауни Джеймс переехал в дом, принадлежащий родителям Рона Макговни, который должен был быть снесён из-за расширения скоростной автомагистрали. Этот дом был идеальным местом для Джеймса и Рона, чтобы собираться вместе, слушать и играть музыку. Джеймс уговорил Рона играть на бас-гитаре и даже обещал преподавать ему.

Третья группа Джеймса, Leather Charm, состояла из бывших членов Phantom Lord. Кроме Джеймса и Рона, в неё вошли Хью Таннер и Джим Муллиган. Группа Leather Charm была более удачной. Она играла свои песни и каверы, например, Quiet Riot Slick Black Cadillac и Remember Tomorrow Iron Maiden. Группа сумела сыграть в нескольких концертах и записала демо, но начала разваливаться. Первым покинул группу Таннер, и его заменил Трой Джеймс. Затем Муллиган ушёл в другую, более прогрессивную группу. В конечном счёте, группа Leather Charm развалилась.

Важнейшим событием в жизни Джеймса стала встреча с выходцем из Дании, барабанщиком Ларсом Ульрихом. Вместе они основали группу Metallica. Поначалу Джеймс решил отказаться от игры на гитаре и только петь, но спустя какое-то время начал играть партии ритм-гитары, а также соло-партии в отдельных песнях.

\section{Несчастные случаи}

Джеймс Хетфилд знаменит несчастными случаями, происходившими с ним. Во время турне в поддержку альбома Master Of Puppets он сломал руку, катаясь на скейтборде. Тогда Джеймса впервые заменил, на тот момент сопровождающий группы, Джон Маршалл. Позже, в 1987 году, он сломал руку ещё раз, в результате чего была отложена запись альбома …And Justice for All, и отменено участие в нескольких шоу предстоящего тура Monsters of Rock’87. Впоследствии в контрактах Джеймса появился пункт «Никаких скейтбордов».

Но, возможно, самым известным является случай на совместном концерте с Guns N’ Roses в 1992 году на Олимпийском стадионе Монреаля. Во время исполнения песни «Fade to Black» Хэтфилд прошёл над пиротехническим оборудованием в тот момент, когда оно сработало. В результате Джеймс получил ожоги левой руки и лица второй и третьей степеней и был вынужден отказаться от игры на гитаре. Вместо ритм-гитариста на время тура был снова приглашен Джон Маршалл из Metal Church. Перед этим концертом пиротехники сообщили группе о том, что во время исполнения «Fade to Black» будут использованы новые спецэффекты: по краям сцены будет фейерверк. Однако пиротехники забыли предупредить, что старые спецэффекты также останутся. Будучи уверенным в том, что ранее заявленных спецэффектов не будет, Джеймс встал рядом с пиротехнической установкой, и столб пламени высотой 4 м обжёг его с левой стороны.

\section{Реабилитация от алкоголизма}

Во время записи альбома St. Anger Джеймс Хэтфилд решил пройти курс реабилитации от алкоголизма. Музыкант сократил своё участие в процессе записи и вернулся в студию для завершения диска через одиннадцать месяцев после курса. Джеймс появился на публике в 2003 году, когда Metallica была номинированна как «MTV Icon» года.

\section{Проблемы с голосовыми связками}

После записи альбома St. Anger в 2003 году Джеймс получил серьёзную травму связок. Из-за того что на этом альбоме использовалась низкая, по сравнению со стандартной, настройка Dropped C, Джеймсу приходилось буквально «выплевывать» слова, что повлекло за собой проблемы с голосом. Постепенно к Джеймсу стал возвращаться голос, что можно услышать, сравнивая более новые концертные записи с записями времён St. Anger. Примерно такая же ситуация возникала в 1991 году, когда Хэтфилд сорвал голос во время записи The Black Album. Тогда на концертах группе пришлось перестраивать гитары на полтона ниже, чтобы Джеймсу было легче петь.

5 февраля 2017 года Metallica должна была выступить в Копенгагене, но у Джеймса снова начались проблемы с голосом, в результате чего концерт был отменен. Три остальных концерта тура Hardwired (3, 7 и 9 февраля в Копенгагене) остались в программе. Концерт, запланированный на 5 февраля, был перенесён на 2 сентября \cite{4}.

\section{Личная жизнь}

Хэтфилд женился на Франческе Томаси 17 августа 1997 года, в браке с ней родилось трое детей: Кэйли (13 июня 1998), Кастор (18 мая 2000) и Марселла (17 января 2002).

В свободное время Джеймс увлекается охотой, коллекционированием редких гитар, разведением пчёл \cite{5}, катанием на скейтборде, сноуборде, водными лыжами, работой в гараже, следит за играми своих любимых команд Oakland Raiders, San Jose Sharks и посещением гонок хот-родов. Джеймсу нравится музыка таких исполнителей, как Black Sabbath, Venom, Motorhead, Lynyrd Skynyrd, Thin Lizzy, Tom Waits, Nick Cave и Ted Nugent[источник не указан 704 дня].

Он выставил свой Chevrolet Camaro 1968 года выпуска на продажу на eBay. Вырученные средства пошли на поддержание музыкальной школьной программы \cite{6}. Автомобиль использовался в клипе на песню «I Disappear» и достался ему в качестве подарка после окончания съемок.

\section{Дискография}

\subsection{Metallica}

\subsubsection{Основная статья:}  \href{https://ru.wikipedia.org/wiki/%D0%94%D0%B8%D1%81%D0%BA%D0%BE%D0%B3%D1%80%D0%B0%D1%84%D0%B8%D1%8F_Metallica}{Дискография Metallica}

\begin{tabular}{|c|c|}
	
\hline 1983 & Kill 'em All \\
\hline 1984 & Ride the Lightning \\
\hline 1986 & Master of Puppets \\
\hline 1988 & …And Justice for All \\
\hline 1991 & Metallica \\
\hline 1996 & Load \\
\hline 1997 & Reload \\
\hline 2003 & St. Anger \\
\hline 2008 & Death Magnetic \\
\hline 2016 & Hardwired...To Self-Destruct \\
\hline
\end{tabular}
 
\section{Гитары}

Джеймс Хэтфилд является обладателем большого количества гитар. В его коллекции есть следующие гитары:

\subsection{Электрогитары}

\begin{enumerate}
	\item Ken Lawrence Explorer
	\item Ken Lawrence Custom Les Paul
	\item Gibson Les Paul 57 Custom
	\item Gibson Les Paul Paul Landers Style
	\item Gibson Explorer 1984
	\item Gibson Explorer 1976 «Rusty»
	\item Gibson Flying V
	\item Zemaitis Flying V
	\item James Trussart SteelDeville
	\item James Trussart SteelX Perforated
	\item ESP Iron Cross
	\item ESP JH-1
	\item ESP JH-2
	\item ESP JH-3
	\item ESP Snakebyte White
	\item ESP Snakebyte BLK
	\item ESP Explorer MX220 «Fuk’em up»
	\item ESP Explorer MX220 «Eet Fuk»
	\item ESP Explorer MX250 Black
	\item ESP Explorer MX250 White
	\item ESP Explorer MX250 «Man to wolf»
	\item ESP Explorer MX250 «Elk»
	\item ESP Explorer MX250 «Deer Skull»
	\item ESP Explorer MX250 «Papa Het»
	\item ESP LTD Grynch
	\item ESP Truckster
	\item ESP Black Truckster
	\item Gretsch White Falcon
	\item Fender Telecaster 1952
	\item Fender Telecaster with B-Bender 
	\item Jackson King V «Kill Bon Jovi»
\end{enumerate}

\subsection{Акустические гитары}

\begin{itemize}
	\item Line 6 Acoustic Variax
	\item Gibson Chet Atkins
\end{itemize}

\begin{thebibliography}{6}	
	\bibitem{1}  Darkos. James Hetfield Solos (24 марта 2017)
	\bibitem{2} Metallica.ru: Биографии
	\bibitem{3} Куда приводит воцерковление из-под палки. Православие и мир (21 октября, 2010)
	\bibitem{4} Концерт 5 февраля перенесён на 2 сентября! (ru-RU)
	\bibitem{5} [PowerfulJRE. Joe Rogan Experience 887 - James Hetfield (16 декабря 2016)
	\bibitem{6} Metallica News. blabbermouth.net (October 23, 2003)	
\end{thebibliography}

\end{document} % конец документа

